% Created 2015-06-13 Sat 08:27
\documentclass[presentation]{beamer}
\usepackage[utf8]{inputenc}
\usepackage[T1]{fontenc}
\usepackage{fixltx2e}
\usepackage{graphicx}
\usepackage{longtable}
\usepackage{float}
\usepackage{wrapfig}
\usepackage{rotating}
\usepackage[normalem]{ulem}
\usepackage{amsmath}
\usepackage{textcomp}
\usepackage{marvosym}
\usepackage{wasysym}
\usepackage{amssymb}
\usepackage{hyperref}
\tolerance=1000
\usetheme{Rochester}
\author{Ustun Ozgur}
\date{2015-06-13 Cumartesi}
\title{React ile AJAX}
\hypersetup{
  pdfkeywords={},
  pdfsubject={},
  pdfcreator={Emacs 24.4.1 (Org mode 8.2.10)}}
\begin{document}

\maketitle

\begin{frame}[label=sec-1]{React ile AJAX}
\begin{itemize}
\item React, Ajax'tan tamamen bağımsızdır
\item Sadece bir görüntü katmanıdır
\item AJAX isteğini yapın, örneğin jQuery get metodu ile
\item Çıktı verişini bileşenin state'inde kaydedin
\begin{itemize}
\item Örneğin success fonksiyonunda setState kullanarak.
\end{itemize}
\end{itemize}
\end{frame}

\begin{frame}[label=sec-2]{componentDidMount}
\begin{itemize}
\item Bu da render gib bir özel yaşamdongusu metodudur
\item Bileşen gerçek DOM'e monte edildiğinde çağrılır
\item Teknik: Bileşen monte olmadan önce state'i başlatın. Monte olduktan sonra
AJAX isteğini başlatın.
\item AJAX sonucu dönünce, bileşen hala monte durumda ise state'i güncelleyin.
\end{itemize}
\end{frame}

\begin{frame}[fragile,label=sec-3]{Örnek: Github}
 \begin{itemize}
\item Amacımız bir kullanıcının depo listesini alıp görüntülemek.
\item Github'in bir ucnoktası var \texttt{https://api.github.com/users/USER\_NAME/repos}
  (buradaki USER$_{\text{NAME}}$ yerine gerçek bir kullanıcı adı gelecek)
\item Bu bir depo dizisi halinde bir json çıktısı döner.
\end{itemize}
\end{frame}

\begin{frame}[fragile,label=sec-4]{getInitialState kullanarak boş depo dizisi}
 \begin{verbatim}
var Repos = React.createClass({
  getInitialState: function () {
    return {repos: []};
  },
  render: function () {
    var renderRepo = function (repo) {
      return <div>Repo {repo.name} {repo.date}</div>
    }
    return <div>
      {this.state.repos.map(renderRepo)}
    </div>

  }

});
\end{verbatim}
\end{frame}

\begin{frame}[fragile,label=sec-5]{Github'dan veriyi çekmek}
 \begin{verbatim}
var GITHUB_ENDPOINT = "https://api.github.com/users/ustun/repos"
var Repos = React.createClass({
  getInitialState: function () { return {repos: []};},

  componentDidMount: function () {
    $.getJSON(GITHUB_ENDPOINT, function (data) {
      this.setState({repos: data});
    }.bind(this));  },
  render: function () {
    var renderRepo = function (repo) {
      return <div>Repo {repo.name} {repo.created_at}</div>}
    return <div>{this.state.repos.map(renderRepo)}</div>
  }
});
\end{verbatim}
\end{frame}

\begin{frame}[fragile,label=sec-6]{Bileşenin hala monte olduğundan emin olmak}
 \begin{itemize}
\item İsteği başlattığımızda bileşen monte
\item AJAX çağrısı dönünce emin değiliz.
\item React şu an monte edilmemiş olan bir bileşenin state'ini değiştiriyor olabilir.
\item Bunu önlemek için state'i monte etmeden önce isMounted kullanın.
\end{itemize}

\begin{verbatim}
componentDidMount: function () {
  $.getJSON(GITHUB_ENDPOINT, function (data) {
    if (this.isMounted()) {
      this.setState({repos: data});
    }
  }.bind(this));  },
\end{verbatim}
\end{frame}

\begin{frame}[label=sec-7]{Özet}
\begin{itemize}
\item AJAX sonuçlarında setState kullanın.
\item AJAX çağrıları genelde componentDidMount metodunda başlatılır.
\item AJAX çağrıları bir olay fonksiyonunda (event handler) da çağrılabilir.
\end{itemize}
\end{frame}

\begin{frame}[label=sec-8]{Alıştırma}
\begin{itemize}
\item Instagram'ın şu adreste bir API'i var
\item \url{https://api.instagram.com/v1/tags/nofilter/media/recent?client_id=CLIENT_ID}
  nofilter herhangi etiketin ismi.
\item Yani
\url{https://api.instagram.com/v1/tags/TAG_NAME/media/recent?client_id=CLIENT_ID}
TAG$_{\text{NAME}}$ doldurulacak
\item Workshop için \url{https://ınstagram.çom/developer/} adresinden bir Instagram
uygulaması oluşturun ve oradaki client id'yi kullanın.
\item HDR etiketli instagramları getirecek bir uygulama yapın.
\item API ilk birkaç sonucu döner, ancak bir de next adında bir parametresi
vardır. Bunu kullanarak bir "Daha Fazla Sonuç Getir" düğmesi yapın.
\item Uyarı (sonraki sayfa)
\end{itemize}
\end{frame}

\begin{frame}[fragile,label=sec-9]{Uyarı}
 \begin{itemize}
\item Instagram same origin policy kullanmaktadır. Bu alıştırma için bunu devre
dışı bırakın
\item \verb~open -a Google\ Chrome --args --disable-web-security~
\item \texttt{chrome.exe -{}-disable-web-security}
\end{itemize}
\end{frame}
% Emacs 24.4.1 (Org mode 8.2.10)
\end{document}
