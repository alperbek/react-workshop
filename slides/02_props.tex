% Created 2015-05-26 Tue 04:20
\documentclass[presentation]{beamer}
\usepackage[utf8]{inputenc}
\usepackage[T1]{fontenc}
\usepackage{fixltx2e}
\usepackage{graphicx}
\usepackage{longtable}
\usepackage{float}
\usepackage{wrapfig}
\usepackage{rotating}
\usepackage[normalem]{ulem}
\usepackage{amsmath}
\usepackage{textcomp}
\usepackage{marvosym}
\usepackage{wasysym}
\usepackage{amssymb}
\usepackage{hyperref}
\tolerance=1000
\usetheme{default}
\usecolortheme{spruce}
\author{Ustun Ozgur}
\date{2015-05-26 Tue}
\title{Props}
\hypersetup{
  pdfkeywords={},
  pdfsubject={},
  pdfcreator={Emacs 24.4.1 (Org mode 8.2.10)}}
\begin{document}

\maketitle

\begin{frame}[fragile,label=sec-1]{Props}
 \begin{itemize}
\item Short for properties
\item Both native components like div, p and our custom components like HelloWorld
can take props
\item Key-value attributes delimited by equals sign
\item To pass \texttt{\{name: "ustun"\}} to HelloWorld component
\item \verb~<HelloWorld name="Ustun"/>~
\end{itemize}
\end{frame}

\begin{frame}[fragile,label=sec-2]{Props}
 \begin{itemize}
\item Can be accessed in the render method and other lifecycle methods
\item Accessed as \texttt{this.props}
\item \texttt{name} key from previous example: \texttt{this.props.name}
\end{itemize}
\begin{verbatim}
var HelloWorld = React.createClass({
  render: function () {
    return <div>{"Hello " + this.props.name}</div>;
  }});
\end{verbatim}
\end{frame}

\begin{frame}[fragile,label=sec-3]{A More Complex Example}
 \begin{itemize}
\item Aim: Greet a number of people
\item People array: \texttt{["John", "Mary", "Susan"]}
\end{itemize}

\begin{verbatim}
var People = React.createClass({
  render: function () {
    var people = ["John", "Mary", "Susan"];
    var peopleNodes = [];

    for (var i = 0; i < people.length; i++) {
      peopleNodes.push(<HelloWorld name={people[i]}/>);
    }
    return <div>{peopleNodes}</div>;
}});
\end{verbatim}
\end{frame}


\begin{frame}[fragile,label=sec-4]{A More Complex Example: Using Map}
 \begin{verbatim}
var People = React.createClass({
  render: function () {
    var people = ["John", "Mary", "Susan"];

    var peopleNodes = people.map(function (person) {
      return <HelloWorld name={person}/>;
    });

    return <div>{peopleNodes}</div>;
}});
\end{verbatim}
\end{frame}

\begin{frame}[fragile,label=sec-5]{Cont'd}
 \begin{itemize}
\item We don't even need the intermediate variable
\item Can directly embed in JSX expression
\end{itemize}

\begin{verbatim}
var People = React.createClass({
  render: function () {
    var people = ["John", "Mary", "Susan"];

    return <div>
      {people.map(function (person) {
        return <HelloWorld name={person}/>; })}
    </div>;
}});
\end{verbatim}
\end{frame}

\begin{frame}[fragile,label=sec-6]{Exercise:}
 \begin{itemize}
\item Modify the todo list application so that a list of todo items (a list of
strings) is passed as props.
\item That is, the ToDos component should be passed a list such as:
\item \texttt{["Buy Tomatoes", "Buy Potatoes"]}
\item Modify the list in code and ensure that the output is updated accordingly.
\item Modify the render method so that the todo items are always sorted
alphabetically.
\end{itemize}
\end{frame}

\begin{frame}[label=sec-7]{Events and Functions as Props}
\begin{itemize}
\item Any JavaScript expression can be passed down as props.
\item Functions in JS are first class values: We can pass functions as props.
\item For event handling, we have props like onClick, onBlur
\item Pass functions for these props.
\end{itemize}
\end{frame}

\begin{frame}[fragile,label=sec-8]{Attaching an onClick handler}
 \begin{verbatim}
var HelloWorld = React.createClass({
  onClick: function () {
    console.log("Hello " + this.props.name);
  },
  render: function () {
    return <div onClick={this.onClick}>
      {"Hello " + this.props.name}
    </div>;
  }
});
\end{verbatim}
\end{frame}

\begin{frame}[fragile,label=sec-9]{People List}
 \begin{itemize}
\item People list and HelloWorld child components
\item Assume that the event handler logic is in People component, not HelloWorld.
\item HelloWorld is just passed an onClick property (event handler)
\item It executes whatever is passed from above.
\end{itemize}

\begin{verbatim}
var HelloWorld = React.createClass({
  render: function () {
    return <div onClick={this.props.onClick}>
      {"Hello " + this.props.name}
    </div>;
  }
});
\end{verbatim}
\end{frame}

\begin{frame}[fragile,label=sec-10]{Complexity in the Parent Component}
 \begin{verbatim}
var People = React.createClass({
  onClick: function (name) {
    console.log("Hello " + name);
  },
  render: function () {
    var people = ["John", "Mary", "Susan"];

    return <div>
      {people.map(function (person) {
        return <HelloWorld
        onClick={this.onClick.bind(this, person)}
        name={person}/>;
      }.bind(this))}
    </div>;
  }});
\end{verbatim}
\end{frame}

\begin{frame}[fragile,label=sec-11]{What Changed?}
 \begin{itemize}
\item We modified onClick handler so that it accepts a name argument.
\end{itemize}
\begin{verbatim}
onClick: function (name) {
\end{verbatim}

\begin{itemize}
\item We customized the onClick handler passed to each HelloWorld by binding the
name parameter to the current person name.
\end{itemize}
\begin{verbatim}
onClick={this.onClick.bind(this, person)}
\end{verbatim}
\end{frame}

\begin{frame}[fragile,label=sec-12]{Bind method}
 \begin{itemize}
\item Introduced in ES5
\item "creates a new function that, when called, has its this keyword set to the
provided value, with a given sequence of arguments preceding any provided
when the new function is called."
\item First purpose: bind the \texttt{this} value
\item Second purpose: bind the arguments: Create \alert{partial} function
\end{itemize}
\end{frame}

\begin{frame}[fragile,label=sec-13]{Example: Function that Adds 5}
 \begin{verbatim}
function add(a, b) { return a + b; }
\end{verbatim}

\begin{itemize}
\item We want to fix \texttt{a} to 5.
\item Value of \texttt{this} not important since unused
\item Bind \texttt{this} to null
\end{itemize}

\begin{verbatim}
var add5 = add.bind(null, 5)
console.log(add5(3)); // 8
\end{verbatim}
\end{frame}


\begin{frame}[fragile,label=sec-14]{Exercise:}
 \begin{itemize}
\item Think about how you can rename the method \texttt{console.log} to \texttt{l}.
\item Does a simple \verb~var l = console.log~ work correctly? What is the correct
solution?
\end{itemize}
\end{frame}

\begin{frame}[fragile,label=sec-15]{Re-visit previous example}
 \begin{verbatim}
var People = React.createClass({
  onClick: function (name) {
    console.log("Hello " + name);
  },
  render: function () {
    var people = ["John", "Mary", "Susan"];
    return <div>
      {people.map(function (person) {
        return <HelloWorld
        onClick={this.onClick.bind(this, person)}
        name={person}/>;
      }.bind(this))}
    </div>;
  }});
\end{verbatim}

\begin{itemize}
\item Second bind: Binding this value.
\item First bind: Binding this value and the additional argument
\end{itemize}
\end{frame}

\begin{frame}[fragile,label=sec-16]{Verbose Version}
 \begin{verbatim}
var People = React.createClass({
  onClick: function (name) {
    console.log("Hello " + name);
  },
  render: function () {
    var people = ["John", "Mary", "Susan"];

    return <div>
    {people.map(function (person) {
      var boundFunction = this.onClick.bind(this, person);
      return <HelloWorld
      onClick={boundFunction}
      name={person}/>;
    }.bind(this))}
    </div>;
}});
\end{verbatim}
\end{frame}

\begin{frame}[fragile,label=sec-17]{Another alternative}
 \begin{itemize}
\item Store the current \texttt{this} value in a variable, for example \texttt{that}
\end{itemize}

\begin{verbatim}
var People = React.createClass({
  onClick: function (name) {
    console.log("Hello " + name);
  },
  render: function () {
    var people = ["John", "Mary", "Susan"];
    var that = this;
    return <div>
      {people.map(function (person) {
        return <HelloWorld
        onClick={that.onClick.bind(that, person)}
        name={person}/>; })}
    </div>;
  }});
\end{verbatim}
\end{frame}


\begin{frame}[fragile,label=sec-18]{Alternative using \_.partial}
 \begin{verbatim}
var People = React.createClass({
  onClick: function (name) {
    console.log("Hello " + name);
  },
  render: function () {
    var people = ["John", "Mary", "Susan"];
    var that = this;
    return <div>
      {people.map(function (person) {
        return <HelloWorld
        onClick={_.partial(that.onClick, person)}
        name={person}/>; })}
    </div>;
  }});
\end{verbatim}
\end{frame}

\begin{frame}[label=sec-19]{getDefaultProps}
\begin{itemize}
\item A component method like render
\item Default property values
\end{itemize}
\end{frame}


\begin{frame}[fragile,label=sec-20]{Example: HelloWorld with Default Greeting}
 \begin{verbatim}
var HelloWorld = React.createClass({
  getDefaultProps: function () {
    return {greeting: 'Hello'}
  },
  render: function () {
    return <div>
      {this.props.greeting} {this.props.name}
    </div>;
  }
});

var People = React.createClass({
  render: function () {
    return <div>
      <HelloWorld name="John"/>
      <HelloWorld greeting="Hola" name="Mary"/>
      </div>
  }});
\end{verbatim}
\end{frame}
% Emacs 24.4.1 (Org mode 8.2.10)
\end{document}
