% Created 2015-06-13 Sat 08:29
\documentclass[presentation]{beamer}
\usepackage[utf8]{inputenc}
\usepackage[T1]{fontenc}
\usepackage{fixltx2e}
\usepackage{graphicx}
\usepackage{longtable}
\usepackage{float}
\usepackage{wrapfig}
\usepackage{rotating}
\usepackage[normalem]{ulem}
\usepackage{amsmath}
\usepackage{textcomp}
\usepackage{marvosym}
\usepackage{wasysym}
\usepackage{amssymb}
\usepackage{hyperref}
\tolerance=1000
\usetheme{Rochester}
\author{Ustun Ozgur}
\date{2015-06-13 Cumartesi}
\title{Props}
\hypersetup{
  pdfkeywords={},
  pdfsubject={},
  pdfcreator={Emacs 24.4.1 (Org mode 8.2.10)}}
\begin{document}

\maketitle

\begin{frame}[fragile,label=sec-1]{Props}
 \begin{itemize}
\item properties (özellikler) için kısaltma
\item Hem div, p gibi hazır bileşenler, hem de MerhabaDünya gibi özel bileşenler,
props alabilir
\item Key-value (anahtar-değer) özellikleri
\item Eşittir işareti
\item MerhabaDunya'ya \texttt{\{ad: "ustun"\}} geçirmek için
\item \verb~<MerhabaDunya ad="Ustun"/>~
\end{itemize}
\end{frame}

\begin{frame}[fragile,label=sec-2]{Props}
 \begin{itemize}
\item Render metodunda ve diğer yaşam döngüsü metodlarında erişilebilir
\item \texttt{this.props} olarak erişilir
\item \texttt{ad} alanına ulaşmak için: \texttt{this.props.ad}
\end{itemize}
\begin{verbatim}
var MerhabaDunya = React.createClass({
  render: function () {
    return <div>{"Merhaba " + this.props.ad}</div>;
  }});
\end{verbatim}
\end{frame}


\begin{frame}[fragile,label=sec-3]{Daha Karmaşık Bir Örnek}
 \begin{itemize}
\item Hedef: Birden fazla insana merhaba de
\item İnsanlar dizisi: \texttt{["Ali", "Ahmet", "Ayse"]}
\end{itemize}

\begin{verbatim}
var Insanlar = React.createClass({
  render: function () {
    var insanlar = ["Ali", "Ahmet", "Ayse"];
    var icerik = [];

    for (var i = 0; i < insanlar.length; i++) {
      icerik.push(<MerhabaDunya ad={insanlar[i]}/>);
    }
    return <div>{içerik}</div>;
}});
\end{verbatim}
\end{frame}

\begin{frame}[fragile,label=sec-4]{Daha Karmaşık Bir Örnek}
 \begin{itemize}
\item \emph{Key} özelliği
\item Her çocuk elemanı eşsiz yapmak için
\item Performans artışı
\end{itemize}

\begin{verbatim}
var Insanlar = React.createClass({
  render: function () {
    var insanlar = ["Ali", "Ahmet", "Ayse"];
    var icerik = [];

    for (var i = 0; i < insanlar.length; i++) {
      icerik.push(<MerhabaDunya key={insanlar[ı]} ad={insanlar[i]}/>);
    }
    return <div>{icerik}</div>;
}});
\end{verbatim}
\end{frame}

\begin{frame}[fragile,label=sec-5]{Daha Karmaşık Bir Örnek: Map Kullanarak}
 \begin{verbatim}
var Insanlar = React.createClass({
  render: function () {
    var insanlar = ["Ali", "Ayşe", "Ahmet"];

    var icerik = insanlar.map(function (person) {
      return <MerhabaDunya key={person} ad={person}/>;
    });

    return <div>{içerik}</div>;
}});
\end{verbatim}
\end{frame}

\begin{frame}[fragile,label=sec-6]{Devam}
 \begin{itemize}
\item Ara değişkenlere ihtiyacımız yok
\item Doğrudan bir JSX ifadesi koyabiliriz
\item Key olarak indeks sayısı kullanabiliriz.
\end{itemize}

\begin{verbatim}
var Insanlar = React.createClass({
  render: function () {
    var insanlar = ["Ali", "Ayse", "Ahmet"];

    return <div>
      {insanlar.map(function (insan, i) {
        return <MerhabaDunya key={i} ad={insan}/>; })}
    </div>;
}});
\end{verbatim}
\end{frame}

\begin{frame}[fragile,label=sec-7]{Alıştırma:}
 \begin{itemize}
\item Todo uygulamasını, yapılacaklar listesi bir string listesi olarak, props
şeklinde girildiği şekilde degistirin.
\item Örnek: ToDos bileşenine şu şekilde bir liste geçirilmelidir:
\item \texttt{["Domates al", "Patates al"]}
\item Listeyi değiştirin, HTML çıktısının değiştiğini görüntüleyin.
\item Render metodunu değiştirerek todo listesinin alfabetik sıralı halini görüntüleyin.
\item Listenin en altında aynı listenin orijinal sırası olacak şekilde bir
değişiklik yapın.
\end{itemize}
\end{frame}


\begin{frame}[label=sec-8]{Olaylar (Events) ve Fonksiyonların Props olarak kullanımı}
\begin{itemize}
\item Herhangi bir JavaScript ifadesi props olarak kullanılabilir.
\item JS'te fonksiyonlar birinci sınıf değerlerdir: Fonksiyonları props olarak
geçirebiliriz.
\item Olay yönetimi için onClick, onBlur gibi adlarda props'larımız vardır.
\item Bu propslar için fonksiyon geçirin.
\end{itemize}
\end{frame}

\begin{frame}[fragile,label=sec-9]{Bir onClick fonksiyonu bağlamak}
 \begin{verbatim}
var MerhabaDunya = React.createClass({
  onClick: function () {
    console.log("Merhaba " + this.props.ad);
  },
  render: function () {
    return <div onClick={this.onClick}>
      {"Merhaba " + this.props.ad}
    </div>;
  }
});
\end{verbatim}
\end{frame}

\begin{frame}[fragile,label=sec-10]{İnsan Listesi}
 \begin{itemize}
\item İnsan listesi ve MerhabaDunya çocuk bileşenleri
\item event handler'ın İnsanlar bileşeninde olduğunu varsayın, MerhabaDünya'da değil
\item MerhabaDunya bileşenine bir onClick özelliği geçirilecek
\item Tepeden ne inerse o çalıştırılacak
\end{itemize}

\begin{verbatim}
var MerhabaDunya = React.createClass({
  render: function () {
    return <div onClick={this.props.onClick}>
      {"Merhaba " + this.props.ad}
    </div>;
  }
});
\end{verbatim}
\end{frame}

\begin{frame}[fragile,label=sec-11]{Karmaşıklık Ana Bileşende}
 \begin{verbatim}
var Insanlar = React.createClass({
  onClick: function (ad) {
    console.log("Merhaba " + ad);
  },
  render: function () {
    var people = ["Ali", "Ahmet", "Ayse"];

    return <div>
      {insanlar.map(function (person, i) {
        return <MerhabaDunya
        key={i}
        onClick={this.onClick.bind(this, insan)}
        ad={insan}/>;
      }.bind(this))}
    </div>;
  }});
\end{verbatim}
\end{frame}

\begin{frame}[fragile,label=sec-12]{Ne Değişti?}
 \begin{itemize}
\item onClick handler'ı ad parametresi alacak şekilde değiştirdik.
\end{itemize}
\begin{verbatim}
onClick: function (ad) {
\end{verbatim}

\begin{itemize}
\item MerhabaDunya'ya geçirilen her onClick handler'ı, ad parametresini şu anki
insan adına bind edecek (bağlamak) şekilde değiştirdik
\end{itemize}
\begin{verbatim}
onClick={this.onClick.bind(this, insan)}
\end{verbatim}
\end{frame}

\begin{frame}[fragile,label=sec-13]{Bind metodu}
 \begin{itemize}
\item ES5 ile geldi
\item Bind = Bağlamak
\item "creates a new function that, when called, has its this keyword set to the
provided value, with a given sequence of arguments preceding any provided
when the new function is called."
\item İlk amaç: \texttt{this} değerini bağlamak
\item İkinci amaç: argümanları bağlayıp \alert{partial} (yarım) bir fonksiyon oluşturmak
\end{itemize}
\end{frame}

\begin{frame}[fragile,label=sec-14]{Örnek: 5 ile Toplama Yapan Fonksiyon}
 \begin{verbatim}
function add(a, b) { return a + b; }
\end{verbatim}

\begin{itemize}
\item \texttt{a}'yı 5'e sabitlemek istiyoruz.
\item \texttt{this}'in değeri önemli değil, çünkü kullanılmıyor
\item \texttt{this} değerini null'a eşitle
\end{itemize}

\begin{verbatim}
var add5 = add.bind(null, 5)
console.log(add5(3)); // 8
\end{verbatim}
\end{frame}


\begin{frame}[fragile,label=sec-15]{Alıştırma:}
 \begin{itemize}
\item \texttt{console.log} metodunu \texttt{l} adında bir fonksiyona nasıl eşitlersiniz?
\item Basit bir \verb~var l = console.log~ çözümü doğru mu? Doğru çözüm ne?
solution?
\end{itemize}
\end{frame}

\begin{frame}[fragile,label=sec-16]{Önceki örneğe tekrar bakış}
 \begin{verbatim}
var Insanlar = React.createClass({
  onClick: function (ad) {
    console.log("Merhaba " + ad);
  },
  render: function () {
    var insanlar = ["Ali", "Ahmet", "Mehmet"];
    return <div>
      {insanlar.map(function (insan, i) {
        return <MerhabaDunya
        key={i}
        onClick={this.onClick.bind(this, insan)}
        ad={insan}/>;
      }.bind(this))}
    </div>;
  }});
\end{verbatim}

\begin{itemize}
\item İkinci bind: this değerini bağlamak.
\item İlk bind: this değerini ve diğer argümanı bağlamak.
\end{itemize}
\end{frame}

\begin{frame}[fragile,label=sec-17]{Daha Uzun Hali}
 \begin{verbatim}
var Insanlar = React.createClass({
  onClick: function (ad) {
    console.log("Merhaba " + ad);
  },
  render: function () {
    var people = ["Ali", "Ahmet", "Mehmet"];

    return <div>
    {insanlar.map(function (insan, i) {
      var boundFunction = this.onClick.bind(this, insan);
      return <MerhabaDunya
      onClick={boundFunction}
       key={i}
      ad={insan}/>;
    }.bind(this))}
    </div>;
}});
\end{verbatim}
\end{frame}

\begin{frame}[fragile,label=sec-18]{Diğer Alternatif}
 \begin{itemize}
\item Şu anki \texttt{this} değerini başka bir değişkende sakla, örneğin \texttt{that}
\end{itemize}

\begin{verbatim}
var Insanlar = React.createClass({
  onClick: function (ad) {
    console.log("Merhaba " + ad);
  },
  render: function () {
    var people = ["Ali", "Ayse", "Ahmet"];
    var that = this;
    return <div>
      {insanlar.map(function (insan, i) {
        return <MerhabaDunya
        key={i}
        onClick={that.onClick.bind(that, insan)}
        ad={insan}/>; })}
    </div>;
  }});
\end{verbatim}
\end{frame}


\begin{frame}[fragile,label=sec-19]{\_.partial kullanarak alternatif}
 \begin{verbatim}
var Insanlar = React.createClass({
  onClick: function (name) {
    console.log("Merhaba " + ad);
  },
  render: function () {
    var people = ["Ali", "Ayse", "Ahmet"];
    var that = this;
    return <div>
      {insanlar.map(function (ad) {
        return <MerhabaDunya
        onClick={_.partial(that.onClick, insan)}
        ad={insan}/>; })}
    </div>;
  }});
\end{verbatim}
\end{frame}

\begin{frame}[label=sec-20]{getDefaultProps}
\begin{itemize}
\item Render gibi bir bileşen metodu
\item Varsayılan özellik değerleri
\end{itemize}
\end{frame}


\begin{frame}[fragile,label=sec-21]{Örnek: Varsayılan Selamlı MerhabaDunya Bileşeni}
 \begin{verbatim}
var MerhabaDunya = React.createClass({
  getDefaultProps: function () {
    return {selam: 'Merhaba'}
  },
  render: function () {
    return <div>
      {this.props.selam} {this.props.ad}
    </div>;
  }
});

var Insanlar = React.createClass({
  render: function () {
    return <div>
      <MerhabaDunya ad="Ali"/>
      <MerhabaDunya selam="Hello" ad="John"/>
      </div>}});
\end{verbatim}
\end{frame}
% Emacs 24.4.1 (Org mode 8.2.10)
\end{document}
