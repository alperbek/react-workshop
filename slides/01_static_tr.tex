% Created 2016-12-16 Fri 05:33
% Intended LaTeX compiler: pdflatex
\documentclass[presentation]{beamer}
\usepackage[utf8]{inputenc}
\usepackage[T1]{fontenc}
\usepackage{graphicx}
\usepackage{grffile}
\usepackage{longtable}
\usepackage{wrapfig}
\usepackage{rotating}
\usepackage[normalem]{ulem}
\usepackage{amsmath}
\usepackage{textcomp}
\usepackage{amssymb}
\usepackage{capt-of}
\usepackage{hyperref}
\usetheme{Rochester}
\author{Üstün Özgür}
\date{2016-12-16 Cuma}
\title{React ile statik sayfa görüntüleme}
\hypersetup{
 pdfauthor={Üstün Özgür},
 pdftitle={React ile statik sayfa görüntüleme},
 pdfkeywords={},
 pdfsubject={},
 pdfcreator={Emacs 25.1.1 (Org mode 9.0)},
 pdflang={English}}
\begin{document}

\maketitle

\begin{frame}[label={sec:org048d91b}]{React ile statik sayfa görüntüleme}
\begin{block}{\emph{render} metodu}
\begin{itemize}
\item render'da bir JSX ifadesi döneriz
\item Sıradan HTML elemanları ya da birleşik (kompozit) React bileşenleri
\end{itemize}
\end{block}
\end{frame}

\begin{frame}[fragile,label={sec:org57d96a0}]{JSX Hakkında}
 \begin{itemize}
\item Bir HTML ya da XML sürümü değil
\item Normal JavaScript'e dönüştürülür
\item \texttt{<div ad="foo" soyad="bar">içerik</div>}
\item \texttt{React.createElement('div', (\{ad: "foo", soyad: "bar"\}, "icerik")}
\item Özellikler \emph{props} adı verilen bir nesnede (object) toplanır
\item Çocuklar sonraki argümanlar olarak geçirilir
\item \texttt{fonksiyon(props, çocuklar)}
\item İç içe gecebilir:: Birleşik (kompozit) bileşenler
\end{itemize}
\end{frame}

\begin{frame}[fragile,label={sec:orgded6bab}]{Daha karmaşık bir örnek}
 \begin{block}{JSX}
\begin{verbatim}
<div ad="foo">
      <span>Bar</span>
      <span>Baz</span>
</div>
\end{verbatim}
\end{block}
\begin{block}{JS}
\begin{verbatim}
React.createElement('div', {ad: 'foo'},
   React.createElement('span', {}, "Bar"),
   React.createElement('span', {}, "Baz"))
\end{verbatim}
\end{block}
\end{frame}

\begin{frame}[label={sec:org2e8a22e}]{JSX: 5 Dakika Verin}
\begin{itemize}
\item JS içinde HTML mi?
\item HTML içinde JS mi?
\item Aslında JS ve HTML karışımı değil, sadece JS
\end{itemize}
\end{frame}

\begin{frame}[fragile,label={sec:org7397930}]{Ana Özellikler}
 \begin{itemize}
\item Sadece tek bir eleman dön
\begin{itemize}
\item Bir fonksiyonun tek bir çıktısı olması gibi
\end{itemize}
\item Bütün elemanlar kapanmalı ya da kendiliğinden kapanır olmalı
\begin{itemize}
\item \texttt{<div></div>}, \texttt{<img/>}, \texttt{<input/>}
\end{itemize}
\item JSX ifadeleri koymak için süslü parantezler
\begin{itemize}
\item \texttt{<div>\{2 + 3\}</div>} Geçerli JSX
\end{itemize}
\end{itemize}
\end{frame}

\begin{frame}[fragile,label={sec:org5cbd744}]{Ana Özellikler}
 \begin{verbatim}
var icTaraf = <div>Merhaba</div>

var disTaraf = <div>{icTaraf} {icTaraf}</div>
\end{verbatim}
\begin{itemize}
\item String birleşimi değil. \texttt{icTaraf} herhangi bir değişken gibi
\end{itemize}
\begin{verbatim}
var icerikler = [];

icerikler.push(icTaraf);
icerikler.push(icTaraf);

var disTaraf = <div>{icerikler}</div>
\end{verbatim}
\end{frame}

\begin{frame}[fragile,label={sec:org0543cfe}]{Ana Özellikler}
 \begin{verbatim}
var isimler = ["Ali", "Ahmet", "Mehmet"];

var icEleman = isimler.map(function (ad) {
    return <div>Merhaba {ad}</div>;
});

var disEleman = <div>{icEleman}</div>
\end{verbatim}
\end{frame}

\begin{frame}[fragile,label={sec:org52b58d1}]{Ana Özellikler}
 \begin{verbatim}
var icerik;

if (this.props.renk == "kirmizi") {
   icerik = <div>Renk kırmızı.</div>
}


dis = <div>{icerik}</div>
\end{verbatim}
\end{frame}

\begin{frame}[fragile,label={sec:orgd36c4e7}]{Ana Özellik}
 \begin{itemize}
\item If ifadeleri statement'tır, expression değildir
\item Ama üç terimliler expression:
\end{itemize}
\begin{verbatim}
a == 3 ? "a 3" : "a 3 değil"
\end{verbatim}

\begin{block}{Örnek}
\begin{verbatim}
outer = <div>{this.props.renk == "kırmızı" ?
		<div>Renk kırmızı</div>:  null}</div>
\end{verbatim}

\begin{verbatim}
outer = <div>{this.props.color == "kırmızı" &&
	       <div>Renk kırmızı</div>}</div>
\end{verbatim}

\begin{itemize}
\item Okumayı zorlaştırabilir
\end{itemize}
\end{block}
\end{frame}

\begin{frame}[fragile,label={sec:orgb7885e7}]{JSX içinde CSS}
 \begin{itemize}
\item JSX aslında arka planda sadece JS
\end{itemize}
\begin{block}{class}
\begin{itemize}
\item \texttt{class} kelimesi JS içinde rezerve (anahtar kelime)
\item Bunun yerine \texttt{className} kullanılır
\item \texttt{<div class="foo bar">} suna dönüşür:  \texttt{<div className="foo bar">}
\end{itemize}
\end{block}
\begin{block}{Stiller}
\begin{itemize}
\item HTML'deki stiller string'dir. CSS'te, JS'teki nesnelere benzer bir sözdizim
kullanılır.
\item JSX'te stiller gerçek JS nesneleridir.
\end{itemize}
\end{block}
\end{frame}

\begin{frame}[fragile,label={sec:org4a5c0b6}]{JSX içinde CSS (devam)}
 \texttt{<div style="color: red; background-color: yellow">}
\begin{itemize}
\item String bir JS nesnesine dönüştürülür
\end{itemize}
\texttt{\{color: 'red', backgroundColor: 'yellow'\}}
\texttt{<div style=\{\{color: 'red', backgroundColor: 'yellow'\}\}}
\begin{itemize}
\item Dikkat: İki tane \texttt{\{}: Birincisi JSX için, ikincisi JS nesnesi için
\item Tire yok: Bunun yerine sonraki harfi büyük harfe çevir
\texttt{background-color} yerine \texttt{backgroundColor}
\end{itemize}
\end{frame}

\begin{frame}[fragile,label={sec:org04a4418}]{JSX içinde CSS (devam)}
 \begin{itemize}
\item JSX'te sayılar: CSS'te çoğu birim px'tır, bu yüzden otomatik olarak eklenir
\end{itemize}
\begin{verbatim}
fontSize:'12px'
\end{verbatim}
ifadesi şu şekilde yazılabilir
\begin{verbatim}
fontSize: 12
\end{verbatim}
\end{frame}

\begin{frame}[fragile,label={sec:org9bb06f2}]{JSX içinde CSS (devam)}
 \begin{itemize}
\item Her zaman beklendiği gibi çalışmaz. Örneğin lineHeight için ana birim 'px'
değil 'em'dir
\end{itemize}
\begin{verbatim}
lineHeight: 18 ifadesi lineHeight: "18em" anlamına gelir
\end{verbatim}

Şunu yazın:
\begin{verbatim}
lineHeight: '18px'
\end{verbatim}
\end{frame}

\begin{frame}[label={sec:org71d6b0a}]{Var olan bir HTML'i React'e çevirmek}
\begin{enumerate}
\item App adında bir React bileşeni oluşturun. Hazır HTML'i alın ve render
metodunda bunu dondurun.

\item Gerekli CSS değişikliklerini yapın. class: className. stiller: nesne

\item Bütün etiketlerin (taglerin) doğru kapandığından emin olun.

\item App.js dosyasını HTML'den çağırın, HTML'e React bileşeninizin
yerleştirileceği ana HTML bileşenini koyun ve React bileşenini `React.render`
ile yerleştirin (mount). (Doğrudan body'ye de yerlestirilebilir.)

\item 2. ve 3. adımlar için dönüştürücüyü (transpiler) izleme (watch) modunda
çalıştırın. Hata vermeyene kadar değişiklikleri yapın.
\end{enumerate}
\end{frame}

\begin{frame}[fragile,label={sec:org4f8dce8}]{JSX'i JavaScript'e Çevirmek İçin Araçlar}
 \begin{itemize}
\item Dönüştürücü (Transpilation)
\item \alert{react-tools}'u şununla yükleyin: \texttt{npm install -g react-tools}
\item İki mod: Tek dosya girişi  ya da klasör girişi
\end{itemize}
\end{frame}

\begin{frame}[fragile,label={sec:orgfe76248}]{Basit Bir Örnek}
 \begin{verbatim}
var MerhabaDunya = React.createClass({
    render: function () {
	    return <div>Merhaba Dunya</div>
    }
})
\end{verbatim}

HTML kısmında ana bileşen için bir içerici (container)
\begin{verbatim}
<div id="app"></div>
\end{verbatim}
Yerleştirmek (monte) için:
\begin{verbatim}
ReactDOM.render(<MerhabaDunya/>, document.getElementById('app'))
\end{verbatim}
\end{frame}

\begin{frame}[label={sec:org210a2c6}]{Alıştırma 1: React ile Başlangıç: 5-10 dakika}
İpucu: Makefile dosyasına bakabilirsiniz.

a - create-react-app ile yeni proje olusturun.

b - 3 farkli yontemle ekrana HelloWorld yazdirin:

React.createClass, function, class HelloWorld

c - JSX kullanmadan React.createElement kullanarak olusturun.
\end{frame}

\begin{frame}[fragile,label={sec:org003dc2d}]{Ozet}
 \begin{itemize}
\item JSX'te JavaScript expressionlarini embed edebiliriz.
\end{itemize}

\begin{verbatim}
function formatName(user) {
  return user.firstName + ' ' + user.lastName;
}

const user = {
  firstName: 'Harper',
  lastName: 'Perez'
};

const element = (
  <h1>
    Hello, {formatName(user)}!
  </h1>
);

ReactDOM.render(
  element,
  document.getElementById('root')
);
\end{verbatim}
\end{frame}

\begin{frame}[fragile,label={sec:org8d24e0f}]{Daha Karmaşık bir örnek: ToDo Uygulaması}
 \begin{itemize}
\item \texttt{examples/02\_todo\_props/mockup.html} alıp React'e çevirin.
\item Çevrimiçi HTML-JSX dönüştürücü :
\url{https://facebook.github.io/react/html-jsx.html}
\item Komut satırı versiyonu: \url{https://www.npmjs.com/package/htmltojsx}
\end{itemize}
\end{frame}


\begin{frame}[fragile,label={sec:org4b9cd72}]{Alıştırma 2:}
 a - \texttt{02\_convert\_mockup/mockup.html}'da verilen HTML'i React'e çevirin. Dikkat:
Burada iki farklı ana kısım var, o yüzden iki ayrı kök eleman (node) olacak.

b - ToDo uygulamasını küçük bileşenlere parçalayın. Şu adlarda bileşenler
olusturun: SearchBar, Todos, TodoItem, Footer.
\end{frame}
\end{document}
